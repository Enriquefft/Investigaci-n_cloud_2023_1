\documentclass{article}

\usepackage{siunitx}
\usepackage{authblk}
\usepackage{biblatex}
\usepackage{booktabs, multirow} % for borders and merged ranges
% \usepackage{soul}% for underlines
\usepackage[table]{xcolor} % for cell colors
% \usepackage{changepage,threeparttable} % for wide tables
\addbibresource{bibliography.bib}


\newcommand{\usd}[1]{\SI{#1}[\$\ensuremath{\,}]{}}


\title{Evaluación de costos de maquinas virtuales}

\date{\today}

\author{
    ********************************* \\
    ********************************* \\
    Enrique Francisco Flores Teniente
  }

\affil{UTEC}

\begin{document}
\maketitle

En el presente informe, vamos a analizar y comparar los precios de los proveedores lideres de cloud computing.

\tableofcontents
\newpage

\section{Introducción}

La Organización de las naciones Unidas (ONI) desea
procesar los datos de censos de población y vivienda de
cada uno de sus 193 países miembro para poder establecer
un mapa de pobreza mundial y poder brindar ayuda de
salud y humanitaria en el futuro. Para ello, requieren de multiples maquinas virtuales capaces de procesar los datos de cada país durante un periodo de 45 días.
Los requisitos minimos son los siguientes:

\subsubsection*{100 maquinas para realizar calculos}
  \begin{itemize}
    \item 2 CPU
    \item 2 GB de RAM
    \item 20 GB de almacenamiento (SSD)
  \end{itemize}

\subsubsection*{maquinas para la base de datos:}
  \begin{itemize}
    \item 4 CPU
    \item 16 GB de RAM
    \item 750 GB de almacenamiento (SSD)
  \end{itemize}

Nuestro objetivo es encontrar el proveedor que nos ofrezca el mejor precio para el procesamiento de los datos.

\section{Criterio de región}

\subsection{Requerimientos}
    
Para elegir la region adecuada nos basamos en 4 criterios \cite{choosing_region}
  \subsubsection*{Politicas}
    La maquina virtual debe estar situada en America para cumplir con la política de privacidad de datos.
  \subsubsection*{Costos}
    Este criterio sera el principal determinante, por lo que haremos una comparacion de precios.
  \subsubsection*{Latencia}
    No sabemos desde donde se realizara la investigación, por lo que podemos ignorar este criterio. Unicamente intentaremos que todas las maquinas virtuales esten en la misma region.
  \subsubsection*{Servicios}
    Durante el proyecto se necesitara una gran potencia de cómputo, por lo que se debe elegir una region que tenga disponible algun servicio con esta caracteristica.

  \subsection{Proveedores}

  \subsubsection*{Amazon Web Services (AWS)}

% \usepackage{booktabs, multirow} % for borders and merged ranges
% \usepackage{soul}% for underlines
% \usepackage[table]{xcolor} % for cell colors
% \usepackage{changepage,threeparttable} % for wide tables
%If the table is too wide, replace \begin{table}[!htp]...\end{table} with
%\begin{adjustwidth}{-2.5 cm}{-2.5 cm}\centering\begin{threeparttable}[!htb]...\end{threeparttable}\end{adjustwidth}
\begin{table}[!htp]\centering

  \caption{Precios MV computo}\label{tab: }
  \scriptsize
  \begin{tabular}{lrrrrrrr}

  \toprule
  Region &Instancia &vCPU &memoria &Network &Costo Mensual &Costo total \\
  \midrule
  \multirow{3}{*}{us-east} &t4g.small &2 &2 GiB &Up to 5 Gigabit &1,226.40 &2,452.80 \\
  &c4.large &2 &3.75 GiB &Moderate &7,300.00 &14,600.00 \\
  &t2.medium &2 &4 GiB &Low to Moderate &3,387.20 &6,774.40 \\
  \midrule
  \multirow{3}{*}{Canada} &t4g.small &2 &2 GiB &Up to 5 Gigabit &1,343.20 &2,686.40 \\
  &c4.large &2 &3.75 GiB &Moderate &8,030.00 &16,060.00 \\
  &t2.medium &2 &4 GiB &Low to Moderate &3,737.60 &7,475.20 \\
  \midrule
  \multirow{3}{*}{us-west} &t4g.small &2 &2 GiB &Up to 5 Gigabit &1,460.00 &2,920.00 \\
  &c4.large &2 &3.75 GiB &Moderate &9,052.00 &18,104.00 \\
  &t2.medium &2 &4 GiB &Low to Moderate &4,029.60 &8,059.20 \\
  \midrule
  \multirow{3}{*}{Sao-Paulo} &t4g.small &2 &2 GiB &Up to 5 Gigabit &1,956.40 &3,912.80 \\
  &c4.large &2 &3.75 GiB &Moderate &11,315.00 &22,630.00 \\
  &t2.medium &2 &4 GiB &Low to Moderate &5,431.20 &10,862.40 \\
  \bottomrule

  \label{table:aws_computo}
  \end{tabular}
\end{table}

\begin{table}[!htp]\centering
\caption{Precios base de datos}\label{tab: }
\scriptsize
\begin{tabular}{lrrrrrrrr}\toprule
Region &Instancia &vCPU &memoria &almacenamiento &Network &Costo Mensual &Costo total \\\midrule
\multirow{3}{*}{us-east} &d3en.xlarge &4 &16 GiB &Up to 25 Gigabit &2 x 14000 HDD &7,679.60 &15,359.20 \\
&m5a.xlarge &4 &16 GiB &Up to 10 Gigabit &EBS only &2,511.20 &5,022.40 \\
&t2.xlarge &4 &16 GiB &Moderate &EBS only &2,709.76 &5,419.52 \\
  \midrule
\multirow{3}{*}{us-west} &d3en.xlarge &4 &16 GiB &Up to 25 Gigabit &2 x 14000 HDD &7,679.60 &15,359.20 \\
&m5a.xlarge &4 &16 GiB &Up to 10 Gigabit &EBS only &2,959.02 &5,918.04 \\
&t2.xlarge &4 &16 GiB &Moderate &EBS only &3,223.68 &6,447.36 \\
  \midrule
\multirow{3}{*}{Canada} &\cellcolor[HTML]{A8A8A8} &\cellcolor[HTML]{A8A8A8} &\cellcolor[HTML]{A8A8A8} &\cellcolor[HTML]{A8A8A8} &\cellcolor[HTML]{A8A8A8} &\cellcolor[HTML]{A8A8A8} &\cellcolor[HTML]{A8A8A8} \\
&m5a.xlarge &4 &16 GiB &Up to 10 Gigabit &EBS only &2,803.20 &5,606.40 \\
&t2.xlarge &4 &16 GiB &Moderate &EBS only &2,990.08 &5,980.16 \\
  \midrule
\multirow{3}{*}{Sao-Paulo} &\cellcolor[HTML]{A8A8A8} &\cellcolor[HTML]{A8A8A8} &\cellcolor[HTML]{A8A8A8} &\cellcolor[HTML]{A8A8A8} &\cellcolor[HTML]{A8A8A8} &\cellcolor[HTML]{A8A8A8} &\cellcolor[HTML]{A8A8A8} \\
&m5a.xlarge &4 &16 GiB &Up to 10 Gigabit &EBS only &4,092.60 &8,185.20 \\
&t2.xlarge &4 &16 GiB &Moderate &EBS only &4,344.96 &8,689.92 \\
\bottomrule
\end{tabular}
\end{table}

  
  \subsubsection*{Microsoft Azure}
  \subsubsection*{Google Cloud Platform (GCP)}

\section{Elección de maquina virtual}
  \subsection{Amazon Web Services (AWS)}

    En nuestra comparación de precios por región evaluamos 3 tipos de instancia.
      \begin{table}[!htp]\centering
        \begin{tabular}{|c|c|c|}

          T4g	& cheapest          \\
          c4	& compute optimized \\
          t2	& general use       \\
          
        \end{tabular}
      \end{table}
    Decidimos elegir estas 3 instancias en función de la familia a la que pertenecen, ya que cada una de estas familias tiene un propósito diferente.
    En primer lugar, la familia T4g es la más barata, pero tiene un rendimiento menor que las otras dos familias. Por otro lado, la familia c4 es la familia de instancias optimizadas para el cómputo, por lo que tiene un rendimiento superior a las otras dos familias, pero también es más cara. Por último, la familia t2 es la familia de uso general, por lo que tiene un rendimiento medio, pero también es más barata que la familia c4.

    En cuanto al almacenamiento, todas las maquinas virtuales de EC2 cuentan con Elastic Block Storage (EBS), el cual, para los requisitos establecidos, tiene un coste de \usd{748.73}.
    En conclusión, para la m

  \subsection{Microsoft Azure}
  \subsection{Google Cloud Platform (GCP)}

\section{Comparación de precios}
  \subsection{Amazon Web Services (AWS)}
  \subsection{Microsoft Azure}
  \subsection{Google Cloud Platform (GCP)}

\section{Recomendación final}

\section{Imágenes, esquemas y diagramas comparativos}


\printbibliography

\end{document}
