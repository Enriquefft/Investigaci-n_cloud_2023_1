\documentclass{article}

\usepackage{siunitx}
\usepackage{authblk}
\usepackage{biblatex}
\usepackage{booktabs, multirow} % for borders and merged ranges
\usepackage{soul}% for underlines
\usepackage[table]{xcolor} % for cell colors
\usepackage{changepage,threeparttable} % for wide tables

% figure
\usepackage{graphicx}

\addbibresource{bibliography.bib}

\newcommand{\usd}[1]{\SI{#1}[\$\ensuremath{\,}]{}}

\title{Prueba de Latex}

\date{\today}

\author{
    Adrian Antonio Auqui Perez \\
    Diego Bustamante Palomino \\
    Enrique Francisco Flores Teniente
  }

\affil{UTEC}

\begin{document}
\maketitle

\section{Regiones y Zonas de disponibilidad}

\subsection{Amazon Web Services (AWS)}

\begin{table}[!htp]\centering
\caption{Regiones AWS}\label{tab: }
\scriptsize
\begin{tabular}{lrrr}\toprule
Continente &Regiones &Zonas de disponibilidad \\\midrule
\multirow{7}{*}{America del Norte} &Óregon &7 \\
&Virginia &10 \\
&California &3 \\
&Ohio &3 \\
&Canada &3 \\
&Govcloud 1 &3 \\
&Govcloud 2 &3 \\
America del sur &Sao Paulo &3 \\
\multirow{8}{*}{Europa} &Irlanda &3 \\
&\textbf{Fráncfort} &3 \\
&Londres &3 \\
&Paris &3 \\
&Estocolmo &3 \\
&Milán &3 \\
&Zúrich &3 \\
&España &3 \\
\multirow{2}{*}{Medio Oriente} &Baréin &3 \\
&EAU &3 \\
Africa &Ciudad del cabo &3 \\
\multirow{10}{*}{Asia} &Singapur &3 \\
&Tokio &4 \\
&Seúl &4 \\
&Bombay &3 \\
&Hong Kong &3 \\
&Osaka &3 \\
&Yakarta &3 \\
&Hyderabad &3 \\
&Pekin &3 \\
&Ningxia &3 \\
\multirow{2}{*}{Australia} &Sidney &3 \\
&Melbourne &3 \\
\bottomrule
\end{tabular}
\end{table}

\newpage

  \subsection{Microsoft Azure}

  \begin{table}[!htp]\centering
\caption{Regiones Microsoft Azure}\label{tab:azure_regions}
\scriptsize
\begin{tabular}{lrrr}\toprule
\multicolumn{2}{c}{Region} &Cantidad de puntos disponibles \\\midrule
\multicolumn{2}{c}{Estados Unidos} &9 \\
\multicolumn{2}{c}{Reino Unido} &2 \\
\multicolumn{2}{c}{Emiratos Arabes Unidos} &2 \\
\multicolumn{2}{c}{Suiza} &2 \\
\multicolumn{2}{c}{Suecia} &2 \\
\multicolumn{2}{c}{Catar} &1 \\
\multicolumn{2}{c}{Noruega} &2 \\
\multicolumn{2}{c}{Corea del Sur} &2 \\
\multicolumn{2}{c}{Japón} &2 \\
\multicolumn{2}{c}{India} &3 \\
\multicolumn{2}{c}{Alemania} &1 \\
\multicolumn{2}{c}{Francia} &2 \\
\multicolumn{2}{c}{Europa} &2 \\
\multicolumn{2}{c}{Canadá} &2 \\
\multicolumn{2}{c}{Brasil} &3 \\
\multicolumn{2}{c}{Azure Government} &3 \\
\multicolumn{2}{c}{Australia} &4 \\
\multicolumn{2}{c}{Asia Pacifico} &2 \\
\multicolumn{2}{c}{África} &2 \\
\bottomrule
\end{tabular}
\end{table}


  \subsection{Google Cloud Platform (GCP)}

\begin{table}[!htp]\centering
  \caption{Regiones y Zonas de disponibilidad\cite{google_regions}}
  \label{tab:regiones}
  \begin{tabular}{|c|c|c|c|c|c|}
    \toprule
    America del Norte & America del Sur & Europa & Asia & Oceania & Africa \\
    3 & 1 & 8 & 7 & 1 & 1 \\
    \bottomrule
  \end{tabular}
\end{table}
 % imagen servidores_google.jpg
 \begin{figure}[!htp]
  \centering
  \includegraphics[width=0.5\textwidth]{servidores_google.jpg}
  \caption{Servidores de Google Cloud Platform}
  \label{fig:servidores_google}
  \end{figure}


\newpage

\section{Precios para Máquinas Virtuales}
% Compute (CPU, Memoria RAM), Storage (Almacenamiento: Disco) y Data Out (Transferencia de Datos Saliente)}

Para la comparación de precios de los servicios de computación, almacenamiento y transferencia de datos salientes, se ha seleccionado la siguiente configuración para cada uno de los proveedores:

\begin{table}[!htp]
  \centering
  \caption{Configuración de las máquinas virtuales}
  \label{tab:configuracion}
  \begin{tabular}{|l|l|}
    \toprule
    Ubicación & EU-Este \\
    Instancias & 1 \\
    Sistema Operativo & Ubuntu 18.04 \\
    Tenancy & Multi-tenant \\
    \bottomrule

  \end{tabular}
\end{table}

Además, se analizaran los precios de 2 maquinas virtuales con diferentes configuraciones de CPU y RAM.
La primera maquina virtual tendrá 2 vCPU y 8 GB de RAM, mientras que la segunda tendrá 64 vCPU y 512 GB de RAM.



  \subsection{Amazon Web Services (AWS)}


  \begin{table}[!htp]\centering
\caption{Maquina basica}\label{tab:MV_Basica_AWS}
\scriptsize
\begin{tabular}{lrrrrr}\toprule
Instancia &vCPU &memoria &Network &Costo mensual\\\midrule
  t4g.large &2 &8 GiB &Up to 5 Gigabit &\usd{0.07} \\
  t2.large &2 &8 GiB &Low to Moderate &\usd{67.74} \\
  m7g.large &2 &8 GiB &Up to 12500 Megabit &\usd{59.57} \\
  m5.large &2 &8 GiB &Up to 10 Gigabit &\usd{70.08} \\
\bottomrule
\end{tabular}
\end{table}

\begin{table}[!htp]\centering
\caption{Maquina potente}\label{tab:MV_Potente_AWS}
\scriptsize
\begin{tabular}{lrrrrr}\toprule
Instancia &vCPU &memoria &Network &Costo mensual\\\midrule
  c6g.16xlarge &64 &128 GiB &25 Gigabit & \usd{1588.48} \\
  m6g.metal &64 &128 GiB &20 Gigabit &\usd{1798.72} \\
  m7g.metal &64 &256 GiB &30 Gigabit &\usd{1906.18} \\
\bottomrule
\end{tabular}
\end{table}

Para esta comparación, decidimos utilizar los siguientes tipos de isntancias: t4g, t2, m7g, m5, c6g, m6g, m7g.
\begin{itemize}
  \item La instancia t4g es una instancia de muy bajo costo, la cual es ideal para pruebas y desarrollo.
  \item La instancia c6g es una instancia de alto rendimiento, la cual es ideal para aplicaciones que requieren computo.
  \item El resto de las instancias son de uso general, es decir, no se especializan en ningun tipo de aplicacion.
\end{itemize} 
\newpage

En cuanto a los precios de almacenamiento AWS, este utiliza EBS (Elastic Block Store) para almacenar los datos de las instancias, y S3 para almacenar datos de forma independiente.
Para comparar, hemos elegido 1 sola instancia de 20GB en SSD, el precio de esta es de \usd{7.49}.
Por otro lado, para el data out, los SSDs de uso general tienen un costo de \usd{0.07} por GB.

  \subsection{Microsoft Azure}

  \begin{table}[!htp]\centering
\caption{Generated by Spread-LaTeX}\label{tab:MV_Azure}
\scriptsize
\begin{tabular}{lrrrrrr}\toprule
\cellcolor[HTML]{A8A8A8}Instancias &\cellcolor[HTML]{A8A8A8}vCPU &\cellcolor[HTML]{A8A8A8}RAM &\cellcolor[HTML]{A8A8A8} &\cellcolor[HTML]{A8A8A8}STORAGE &\cellcolor[HTML]{A8A8A8}Precio/mes (USO) \\\midrule
  \cellcolor[HTML]{A8A8A8}B2ms &2 &4GB & &16GB &\usd{60.736}/mes \\
  \cellcolor[HTML]{A8A8A8}E64as V4 &64 &512GB & &1024GB &\usd{2943.3600}/mes \\
\bottomrule
\end{tabular}
\end{table}

  \subsection{Google Cloud Platform (GCP)}


\begin{table}[!htp]\centering
\caption{Maquina basica}\label{tab:MV_google}
\scriptsize
\begin{tabular}{lrrrrr}\toprule
Instancia &vCPU &memoria &Network &Costo mensual\\\midrule
  e2-custom-2-8192 &2 &8 GiB &1 Gbps &\usd{51.65} \\
  e2-custom-64-524288 &64 &512 GiB &1 Gbps &\usd{2546.95} \\
\bottomrule
\end{tabular}
\end{table}


\section{Comparación entre AWS, Azure y Google}

\begin{table}[!htp]\centering
\caption{Generated by Spread-LaTeX}\label{tab: }
\scriptsize
\begin{tabular}{lrrr}\toprule
Proveedor &Precio &Conexion \\\midrule
AWS &59.57 &12500 \\
Google &51.65 &1800 \\
Microsoft &60.74 &1000 \\
\bottomrule
\end{tabular}
\end{table}

Para cada proveedor, elegimos una maquina con 2 vCPU, 8 de RAM y anotamos su velocidad de conexion a internet.
Con estos datos, podemos comparar los precios de cada proveedor, y asi determinar cual es el proveedor que ofrece el mejor precio por vCPU y por velocidad de conexion a internet.
En conclusión, podemos decir que el mejor proveedor es AWS, ya que ofrece un buen precio por vCPU y una excelente velocidad de conexion a internet.

\section{Imágenes, esquemas y diagramas comparativos}

\begin{figure}[!htp]
  \centering
  \includegraphics[width=0.8\textwidth]{precios_p1.jpg}
  \caption{Precios de las maquinas virtuales}
  \label{fig:precios_p1}
\end{figure}
\begin{figure}[!htp]
  \centering
  \includegraphics[width=0.8\textwidth]{conexion_p1.jpg}
  \caption{Velocidad de conexion a internet}
  \label{fig:conexion_p1}
\end{figure}

\printbibliography

\end{document}
