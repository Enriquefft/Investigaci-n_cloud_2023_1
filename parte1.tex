\documentclass{article}

\usepackage{siunitx}
\usepackage{authblk}
\usepackage{biblatex}
\usepackage{booktabs, multirow} % for borders and merged ranges
\usepackage{soul}% for underlines
\usepackage[table]{xcolor} % for cell colors
\usepackage{changepage,threeparttable} % for wide tables

\addbibresource{bibliography.bib}

\newcommand{\usd}[1]{\SI{#1}[\$\ensuremath{\,}]{}}

\title{Prueba de Latex}

\date{\today}

\author{
    ********************************* \\
    ********************************* \\
    Enrique Francisco Flores Teniente
  }

\affil{UTEC}

\begin{document}
\maketitle

\section{Introducción}



\section{Regiones y Zonas de disponibilidad}

\subsection{Amazon Web Services (AWS)}

\begin{table}[!htp]\centering
\caption{Generated by Spread-LaTeX}\label{tab: }
\scriptsize
\begin{tabular}{lrrr}\toprule
Continente &Regiones &Zonas de disponibilidad \\\midrule
\multirow{7}{*}{America del Norte} &Óregon &7 \\
&Virginia &10 \\
&California &3 \\
&Ohio &3 \\
&Canada &3 \\
&Govcloud 1 &3 \\
&Govcloud 2 &3 \\
America del sur &Sao Paulo &3 \\
\multirow{8}{*}{Europa} &Irlanda &3 \\
&\textbf{Fráncfort} &3 \\
&Londres &3 \\
&Paris &3 \\
&Estocolmo &3 \\
&Milán &3 \\
&Zúrich &3 \\
&España &3 \\
\multirow{2}{*}{Medio Oriente} &Baréin &3 \\
&EAU &3 \\
Africa &Ciudad del cabo &3 \\
\multirow{10}{*}{Asia} &Singapur &3 \\
&Tokio &4 \\
&Seúl &4 \\
&Bombay &3 \\
&Hong Kong &3 \\
&Osaka &3 \\
&Yakarta &3 \\
&Hyderabad &3 \\
&Pekin &3 \\
&Ningxia &3 \\
\multirow{2}{*}{Australia} &Sidney &3 \\
&Melbourne &3 \\
\bottomrule
\end{tabular}
\end{table}

\newpage

  \subsection{Microsoft Azure}
  \subsection{Google Cloud Platform (GCP)}
   


\section{Precios para Máquinas Virtuales}
% Compute (CPU, Memoria RAM), Storage (Almacenamiento: Disco) y Data Out (Transferencia de Datos Saliente)}

Para la comparación de precios de los servicios de computación, almacenamiento y transferencia de datos salientes, se ha seleccionado la siguiente configuración para cada uno de los proveedores:

\begin{table}[!htp]
  \centering
  \caption{Configuración de las máquinas virtuales}
  \label{tab:configuracion}
  \begin{tabular}{|l|l|}
    \toprule
    Ubicación & EU-Este \\
    Instancias & 1 \\
    Sistema Operativo & Ubuntu 18.04 \\
    Tenancy & Multi-tenant \\
    \bottomrule

  \end{tabular}
\end{table}

Además, se analizaran los precios de 2 maquinas virtuales con diferentes configuraciones de CPU y RAM.
La primera maquina virtual tendrá 2 vCPU y 8 GB de RAM, mientras que la segunda tendrá 64 vCPU y 512 GB de RAM.



  \subsection{Amazon Web Services (AWS)}


  \begin{table}[!htp]\centering
\caption{Maquina basica}\label{tab: }
\scriptsize
\begin{tabular}{lrrrrr}\toprule
Instancia &vCPU &memoria &Network &Costo \\\midrule
t4g.large &2 &8 GiB &Up to 5 Gigabit &0.07 \\
t2.large &2 &8 GiB &Low to Moderate &67.74 \\
m7g.large &2 &8 GiB &Up to 12500 Megabit &59.57 \\
m5.large &2 &8 GiB &Up to 10 Gigabit &70.08 \\
\bottomrule
\end{tabular}
\end{table}

\begin{table}[!htp]\centering
\caption{Maquina potente}\label{tab: }
\scriptsize
\begin{tabular}{lrrrrr}\toprule
Instancia &vCPU &memoria &Network &Costo \\\midrule
c6g.16xlarge &64 &128 GiB &25 Gigabit &1,588.48 \\
m6g.metal &64 &128 GiB &20 Gigabit &1,798.72 \\
m7g.metal &64 &256 GiB &30 Gigabit &1,906.18 \\
\bottomrule
\end{tabular}
\end{table}



  \subsection{Microsoft Azure}
  \subsection{Google Cloud Platform (GCP)}

\section{Comparación entre AWS, Azure y Google}

\section{Imágenes, esquemas y diagramas comparativos}


\printbibliography

\end{document}
